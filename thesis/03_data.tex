\section{Dane}
\subsection{Zbiór danych}
Wykorzystany zbiór danych pozyskano z serwisu \url{codeocean.com} \cite{dane}. Zbiór ten udostępniona na licencji GNU General Public License (GPL) a jego autorami są: 

\begin{enumerate}
    \item Paola A. Sánchez-Sánchez
    \item José Rafael García-González
    \item Juan Manuel Rua Ascar. 
\end{enumerate}

\noindent
Cała trójka z pochodzi Universidad Simón Bolívar, Barranquila w Kolumbii.\\

\noindent
Zbiór zawierał anonimowe dane 400 rozpoznanych przypadków a każdy z przypadków 23 cechy. Cechy miały różny typ (np. wiek pacjenta (typ całkowity) czy wystąpienie danego objawu (typ binarny)) co przemawiało za użyciem normalizacji przy użyciu MinMaxScalera z biblioteki Scikit-learn.\\

\noindent
W zbiorze znajdowały się dane dotyczące 7 rodzajów bólu głowy.
Zbiór nie był zbiorem zbalansowanym (co należy mieć na uwadze w dalszej analizie):

\begin{verbatim}
    Type
    Basilar-type aura 		            18
    Familial hemiplegic migraine   24
    Migraine without aura 		        60
    Other 				                        17
    Sporadic hemiplegic migraine   14
    Typical aura with migraine 	  247
    Typical aura without migraine  20
    dtype: int64
\end{verbatim}

\noindent
Zbiór nie posiadał brakujących danych więc nie zaistniała konieczność imputacji.

\subsection{Informacje prawne}
\noindent
Zbiór udostępniony został na licencji GNU General Public License (GPL) \cite{dane}.\\

\noindent
Wykorzystane oprogramowanie korzystało z licencji:\\
\indent Język Python: Python Software Foundation License \cite{python}\\
\indent Biblioteka Pandas: BSD 3-Clause License \cite{pandas}\\
\indent Biblioteka NumPy: BSD 3-Clause License \cite{numpy}\\
\indent Biblioteka Seaborn: BSD 3-Clause License \cite{seaborn}\\
\indent Biblioteka TensorFlow: Apache License 2.0 \cite{tensorflow}\\

\noindent
Wspomniane biblioteki zostały szczegołowo opisane w następujących publikacjach naukowych:\\
\indent Język Python \cite{python_paper}\\
\indent Pandas: \url{https://zenodo.org/records/10957263} \cite{pandas_paper}\\
\indent NumPy: \url{https://www.nature.com/articles/s41586-020-2649-2} \cite{numpy_paper}\\
\indent Seaborn: \url{https://joss.theoj.org/papers/10.21105/joss.03021} \cite{seaborn_paper}\\
\indent TensorFlow: \url{https://zenodo.org/records/10798587} \cite{tensorflow_paper}\\

