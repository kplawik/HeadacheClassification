\section{Wstęp}
Bóle głowy bywają trudne do sklasyfikowania. O ile z obserwacji własnych miałem niestety okazję się o tym przekonać to nawet i świat nauki od lat również boryka się z tym problemem. Brytyjski instytut znany jako Headache Classification Committee of the International Headache Society (IHS) rozróżnia 13 kategorii bólów głowy - a samej tylko migreny - 29 typów \cite{trudnosc}. Co więcej instytut ten wyraźnie mówi o tym że pacjent może cierpieć na więcej niż jeden z rodzaj (\cite{trudnosc} punkt 9 we wstępie). Badania przeprowadzone przez EHF (European Headache Federation) \cite{ehf} również potwierdzają że dominujący ból głowy nie musi być jedynym \cite{kilka}.\\

W pomocą przychodzi nam zagadnienie Uczenia Maszynowego oraz powiązane z nim algorytmy klasyfikacyjne. Poniższa praca dokumentuje wyniki kilkudziesięciu eksperymentów mających na celu automatyczną klasyfikację przy użyciu zarówno algorytmów regresyjnych (np. kNN) jak i głębokich Sieci Neuronowych (Deep Learning).